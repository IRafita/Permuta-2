\documentclass{article}
\usepackage[utf8]{inputenc}

\begin{document}
Si ordenem per exemple la llista 0 1 1 2
\begin{enumerate}
\setcounter{enumi}{-1}
\item 0 1 1 2
\item 0 1 2 1
\item 0 2 1 1
\item 1 0 1 2
\item 1 0 2 1
\item 1 1 0 2
\item 1 1 2 0
\item 1 2 0 1
\item 1 2 1 0
\item 2 0 1 1
\item 2 1 0 1
\item 2 1 1 0
\end{enumerate}

Ara volem treure el 7
\begin{itemize}
\item T = 7
\item l = 0 1 1 2
\item Pos = 12
\item elements = 4
	\begin{itemize}
	\item tp = (T*elements)/Pos
	\item b = BuscarMesPetit (tp) = Buscar (2)
		\subitem b = B(1) = 1
	\item o = * * * 1
	\item T -= tp * b = 3*1 -> 4
	\item Pos *= repe/elements = 2/4 -> 6
	\end{itemize}

\item T = 4
\item elements = 3
\item l = 0 1 2
\item Pos = 6
	\begin{itemize}
	\item tp = (T*elements)/Pos = (4*3)/6 = 2
	\item b = BuscarMesPetit (tp) = 2
	\item o = * * 2 1
	\item T -= tp * b = 2*2 -> 0
	\item o = 1 0 2 1
	\end{itemize}
\end{itemize}

Suposem que no se pot fer de petit a gran, ho almenys això està fora del nostre coneixement amb el plantejament que fem.
$$T -\!\!\!= \frac{T \times elements \times b}{Pos}$$
On b és on se troba el seu representant mínim.\\

En conceqüència d'aquest, veiem que seria preferible fer canvis als programes anteriors per a evitar fer Parsers per poder-ho fer compatible
\begin{itemize}
\item next, fent que sigui al reves (m'explico, que el petit sigui ara el gran)
\item Vector digit, fent que el sort, el faci del reves, ja que aixi no funcionara massa be
\end{itemize}

Gràcies a aquest algoritme, normalment podrem fer sense problemes obtimitzar molt d'espai
\end{document}
