\documentclass{article}
\usepackage[utf8]{inputenc}

\begin{document}
Si ordenem per exemple la llista 0 1 1 2
\begin{enumerate}
\setcounter{enumi}{-1}
\item 0 1 1 2
\item 0 1 2 1
\item 0 2 1 1
\item 1 0 1 2
\item 1 0 2 1
\item 1 1 0 2
\item 1 1 2 0
\item 1 2 0 1
\item 1 2 1 0
\item 2 0 1 1
\item 2 1 0 1
\item 2 1 1 0
\end{enumerate}

Ara sabem que 1 2 0 1 és el 7, llavors com podem treurel directament?
\begin{itemize}
\item 1 2 0 1
	\begin{itemize}
	\item read = 0
	\item list = {1}
	\item Pos = 2
	\item repe = 1
	\item elements = 2
		\subitem b = 0
		\subitem D = 0
	\end{itemize}

\item 1 2 1 0
	\begin{itemize}
	\item read = 2
	\item list = {1, 0}
	\item Pos = 6
	\item repe = 1
	\item elements = 3
		\subitem b = 2
		\subitem D = 4
	\end{itemize}

\item 1 2 1 0
	\begin{itemize}
	\item read = 1
	\item list = {2, 1, 0}
	\item Pos = 12
	\item repe = 2
	\item elements = 4
		\subitem b = 1
		\subitem D = 7
	\end{itemize}
\end{itemize}
Preferiblement de Petit a gran, ja que així el valor de Possiblitats és més fàcil de calcular.
$$D +\!\!\!= \frac{Pos \times b}{elements}$$
On b és on se troba.
\end{document}
